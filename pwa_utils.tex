\documentclass[a4paper, 12pt]{article}
\usepackage[utf8]{inputenc}
\usepackage{hyperref}
\usepackage{fullpage}
\title{Pi-web-agent utilities}


\begin{document}
\maketitle
\section{Automatic update of files during development}

Running \texttt{inter\_update.py} script while developing for the pi-web-agent will automatically copy your changes to the correct place everytime you save a source file. For now source files are: \texttt{*.py, *.htm, *.js}. The script is located in \texttt{usr/bin} under the root of the pi-web-agent directory structure. \\


\textit{Synopsis:}
\begin{itemize}
\item[] inter\_update [-h] -wd WORK\_DIR
\end{itemize}

\textit{Arguments:}
\begin{itemize}
\item[] -wd, --work-dir=WORK\_DIR
\end{itemize}

\textit{Example usage:}
\begin{itemize}
\item[] \texttt{sudo usr/bin/inter\_update -wd /home/rpi/pi-web-agent/}
\end{itemize}


\textit{Requirements:}
\begin{itemize}
\item[] pyinotify
\end{itemize}

Notice that it will probably need sudo access to be able to copy the files under \texttt{/usr}. Also, with the move to C for some of the extensions this should be extended so that it will compile and then copy.
 
\section{Testing framework for extensions}
\end{document}







\end{document}